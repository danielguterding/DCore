\documentclass[disablejfam,12pt]{article}
\usepackage{graphicx}
\usepackage{color}
\pagestyle{empty}
\setlength{\textheight}{240mm}
\setlength{\textwidth}{160mm}
\setlength{\columnsep}{5mm}
\setlength{\topmargin}{-30mm}
\setlength{\oddsidemargin}{0mm}
\setlength{\evensidemargin}{0mm}
\usepackage{amssymb}
\usepackage{amsmath}
\usepackage{braket}

\newcommand{\mathi}{\ensuremath{\mathrm{i}}}
\newcommand{\mi}{\ensuremath{\mathrm{i}}}
\newcommand{\barT}{\ensuremath{\bar{T}}}

\author{Junya Otsuki}

\begin{document}
\title{ALPSCore/CT-HYB-SEGMENT}
\maketitle
\thispagestyle{empty}

\section{Two-particle Green's function}

\subsection{Definition}
\begin{eqnarray}
 G_{1234}(\tau_1, \tau_2, \tau_3, \tau_4) &\equiv& \braket{T_\tau c_1(\tau_1) c_2^\dagger(\tau_2) c_3(\tau_3) c_4^\dagger(\tau_4)}\\
 G_{1234}(i \omega, i \omega', i\nu) &=& \frac{1}{\beta} \int_0^\beta  d \tau_{1} d \tau_{2} d \tau_{3} d \tau_{4} G_{1234}(\tau_{1}, \tau_{2}, \tau_{3}, \tau_{4}) e^{i\omega \tau_{12}} e^{i\omega' \tau_{34}}e^{i \nu \tau_{14}}
\end{eqnarray}

\subsection{Numerical data}
For density-density interactions, we have two kinds of components
$G_{iijj}(i\omega, i\omega', i\nu)$ and
$G_{ijji}(i\omega, i\omega', i\nu)$.
The program computes only
$G_{iijj}(i\omega, i\omega', i\nu)$ for $i \geq j$ and $\nu \geq 0$.
Other components need to be converted from this data.

\subsection{Conjugate}
By taking the complex conjugate, we obtain
\begin{align}
	G_{1234}(i \omega, i \omega', i\nu) = - G_{4321}(-i \omega', -i \omega, -i\nu)^*,
	\label{eq:conjugate}
\end{align}
which leads to
\begin{align}
	G_{iijj}(i \omega, i \omega', i\nu) &= - G_{jjii}(-i \omega', -i \omega, -i\nu)^*,
	\\
	G_{ijji}(i \omega, i \omega', i\nu) &= - G_{ijji}(-i \omega', -i \omega, -i\nu)^*.
\end{align}
The first equation can be used to extend $G_{iijj}$ into the negative bosonic frequency side.
\emph{But, we cannot obtain $G_{iijj}$ for $i<j$ and $\nu>0$}



\subsection{Single-index interchange}
By interchanging $c_2^{\dag}$ and $c_4^{\dag}$, we obtain
\begin{align}
	G_{1234}(i \omega, i \omega', i\nu) = - G_{1432}(i \omega' + i\nu, i \omega', i\omega - i\omega'),
\end{align}
which leads to
\begin{align}
	G_{iijj}(i \omega, i \omega', i\nu) = - G_{ijji}(i \omega' + i\nu, i \omega', i\omega - i\omega').
\end{align}
This equation can be used to convert $G_{iijj}$ to $G_{ijji}$. We note that finite bosonic frequencies of $G_{iijj}$ are required even if we need only the zero bosonic frequency of $G_{ijji}$.

Similarly, we can obtain another relation by an exchange of $c_1$ and $c_3$
\begin{align}
	G_{1234}(i \omega, i \omega', i\nu) = - G_{3214}(i \omega, i \omega + i\nu, i\omega' - i\omega).
\end{align}
which leads to
\begin{align}
	G_{iijj}(i \omega, i \omega', i\nu) = - G_{jiij}(i \omega, i \omega + i\nu, i\omega' - i\omega).
\end{align}



\subsection{Two-index interchange}
By interchanging $c_1 c_2^{\dag}$ and $c_3 c_4^{\dag}$, we obtain
\begin{align}
	G_{1234}(i \omega, i \omega', i\nu) &= G_{3412}(i \omega' + i\nu, i \omega + i\nu, -i\nu) \\
	&= G_{2143}(-i \omega - i\nu, -i \omega' - i\nu, i\nu)^*.
\end{align}
Here, we used Eq.~(\ref{eq:conjugate}) in the second line.
The first equation leads to
\begin{align}
	G_{iijj}(i \omega, i \omega', i\nu) &= G_{jjii}(i \omega' + i\nu, i \omega + i\nu, -i\nu), \\
	G_{ijji}(i \omega, i \omega', i\nu) &= G_{jiij}(i \omega' + i\nu, i \omega + i\nu, -i\nu).
\end{align}
% \begin{align}
% 	G_{iijj}(i \omega, i \omega', i\nu) = G_{iijj}(-i \omega - i\nu, -i \omega' - i\nu, i\nu)^*,
% 	\\
% 	G_{ijji}(i \omega, i \omega', i\nu) = G_{jiij}(-i \omega - i\nu, -i \omega' - i\nu, i\nu)^*.
% \end{align}



\end{document}
